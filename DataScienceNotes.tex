%        File: 
%     Created: Thu Sep 17 10:00 PM 2015 C
% Last Change: Thu Sep 17 10:00 PM 2015 C
%
\documentclass[11pt,letterpaper,oneside]{memoir}
\usepackage[margin=2cm]{geometry}
\usepackage{verbatim}
\usepackage{thmtools}
\usepackage{amsthm}
\newtheorem{thm}{Theorem}
\newtheorem{lem}{Lemma}
% create environments for definitions and questions
\theoremstyle{definition}
\newtheorem{defn}{Definition}

\usepackage[nogroupskip,acronym]{glossaries}
\makenoidxglossaries
\loadglsentries{DataScienceNotesGA}
\printnoidxglossaries
%\printacronyms
\printglossary[type=\acronymtype]

\usepackage[T1]{fontenc}
\usepackage{newtxtext,newtxmath}
\usepackage{url}
\usepackage[svgnames]{xcolor}
\ifpdf
  \usepackage{pdfcolmk}
\fi
%% check if using xelatex rather than pdflatex
\ifxetex
  \usepackage{fontspec}
\fi
\usepackage{graphicx}
\usepackage{hyperref}
%% drawing package
\usepackage{tikz}
%% for dingbats
\usepackage{pifont}
\providecommand{\HUGE}{\Huge}% if not using memoir
\newlength{\drop}% for my convenience
\newcommand*{\TXfont}[1]{\fontencoding{T1}\fontfamily{#1}\selectfont}
%% Generic publisher’s logo
\newcommand*{\plogo}{\fbox{$\mathcal{YW}$}}

\newcommand{\etal}{\textit{et al}.}
\newcommand{\ie}{\textit{i}.\textit{e}. }
\newcommand{\eg}{\textit{e}.\textit{g}. }

%% Some shades
\definecolor{Dark}{gray}{0.2}
\definecolor{MedDark}{gray}{0.4}
\definecolor{Medium}{gray}{0.6}
\definecolor{Light}{gray}{0.8}

\newcommand*{\rotrt}[1]{\rotatebox{90}{#1}}
\newcommand*{\rotlft}[1]{\rotatebox{-90}{#1}}
\newcommand*{\topb}{%
\resizebox*{\unitlength}{\baselineskip}{\rotrt{$\}$}}}
\newcommand*{\botb}{%
\resizebox*{\unitlength}{\baselineskip}{\rotlft{$\}$}}}
\newcommand*{\titleBC}{\begingroup
\begin{center}
\def\CP{\textit{\HUGE Data Science Notebook}}
\settowidth{\unitlength}{\CP}
{\color{LightGoldenrod}\topb} \\[\baselineskip]
\textcolor{Sienna}{\CP} \\[\baselineskip]
{\color{RosyBrown}\LARGE PRINCIPLES OF DATA SCIENCE} \\
{\color{LightGoldenrod}\botb}
\end{center}
\vfill
\begin{center}
{\LARGE\textbf{Yigal Weinstein}}\\
\vfill
\plogo\\[0.5\baselineskip]
2016
\end{center}
\endgroup}
\makeindex
\begin{document}
\pagestyle{empty}
\titleBC
\clearpage
\chapter*{Prologue}
There is content that I'm trying to learn in the pursuit of data science. This note book serves the following purposes:
\begin{enumerate}
  \item Continuously analyze and synthesize new material as I'm brought into contact with it.  Each chapter is a topic referenced in Zipfian's Data Science Primer \cite{data_science_primer}.
  \item Have a catalog of what I've studied up until entering into the boot camp.
  \item Be honest with the quantity and quality of work I've produced.
\end{enumerate}

\chapter{Machine Learning}
%\begin{description}
%  \item[Task]
%    An abstract representation of a problem we want to solve regarding the domain objects of import.
%  \item[decision rule]
%    given an observable random variable $x$ over the probability space  $(\mathcal{x},\sigma, p_\theta)$, determined by a parameter $\theta \in \theta$, and a set a of possible actions, a (deterministic) decision rule is a function $\delta : \mathcal{x} \rightarrow a$.
%  \item[Domain objects]
%    Elements of an intuitive model prior to applying machine learning to it, \eg emails in a spam filter, movies in a rating system
%  \item[Feature representation]
%    Describe the relevant objects in our domains for use in a machine learning model.
%  \mdef{Machine learning}{
%      \item The systematic study of algorithms and systems that improve their knowledge or performance with experience.
%      \item Using the right features to build the right models that achieve the right tasks
%      }
%  \item[Generalization]
%    being able to use training data as experience to improve on the outcome of a given task
%  \mdef{Model}{
%    \item A map from data points to outputs
%    \item is itself produced as the output of a machine learning algorithm applied to training data
%    }
%  \item[Posterior probability]
%    probability taken after a given feature was observed
%\end{description}
\section{Quotes}

``models lend the machine learning field diversity, but tasks and features give it unity.'' \cite[p. 13]{Flach2012}
Machine learning is all about using the proper algorithms to accomplish the desired \glspl{task}.
\begin{comment}
\begin{defn}[Task]
  \label{defn:task}
  An abstract representation of a problem we want to solve regarding the domain objects of import.
\end{defn}
\end{comment}

\chapter{Python}
There is a Pythonic way of coding.  This is a culturally obtained set of standards.  Values are,

\begin{itemize}
  \item explicit
  \item choose simple over complex
  \item maximize readability
\end{itemize}

\chapter{Algorithms}
\chapter{SQL}
\chapter{Probability}
\chapter{Statistics}
\chapter{Linear Algebra}
\printnoidxglossaries
%\printacronyms
\bibliography{DataScienceNotes}
\bibliographystyle{plain}
\end{document}
\begin{comment}
  \item[Model]
    \mbox{}\\[-1.7\baselineskip]
    \begin{enumerate}
      \item A map from data points to outputs
      \item is itself produced as the output of a machine learning algorithm applied to training data

## Example: SpamAssassin: Spam vs Ham

+ Linear Classifier
* decision rule:
$\mathbf{w}\cdot\mathbf{x} > t$, where the message is deemed spam if the lhs is greater than the rhs
* 
decision boundary (plane): 
$\mathbf{w}\cdot\mathbf{x} = t$
separates spam from ham
* homogeneous equation by addition of constant variable $x_0=1$ where $w_0=-t$, so that the decision rule for example can be written, 
\begin{equation} 
  {\mathbf{w}^\circ} \cdot {\mathbf{x}^\circ} > 0 
\end{equation}
, where  $\mathbf{w}^\circ = (-t, w_1, w_2, ..., w_n)$ and $\mathbf{x}^\circ = (1, x_1, x_2, ..., x_n)$.
+ Overfitting
* Over training for too specialized a case


# Chapter 1

## Quotes


## 1.1 Tasks: the problems that can be solved with machine learning


## Models: 
\begin{thm}
  just a test
\end{thm}

\end{comment}
